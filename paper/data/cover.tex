\thusetup{
  %******************************
  % 注意:
  %   1. 配置里面不要出现空行
  %   2. 不需要的配置信息可以删除
  %******************************
  %
  %=====
  % 秘级
  %=====
  secretlevel={秘密},
  secretyear={10},
  %
  %=========
  % 中文信息
  %=========
  ctitle={威胁情报数据融合与推送服务系统},
  cdegree={工学学士},
  cdepartment={计算机科学与技术系},
  cmajor={计算机科学与技术},
  cauthor={黄锐皓},
  csupervisor={尹霞教授},
  % cassosupervisor={段海新教授}, % 副指导老师
  % ccosupervisor={张甲教授}, % 联合指导老师
  % 日期自动使用当前时间,若需指定按如下方式修改:
  % cdate={超新星纪元},
  %
  % 博士后专有部分
  % cfirstdiscipline={计算机科学与技术},
  % cseconddiscipline={系统结构},
  % postdoctordate={2009年7月——2011年7月},
  % id={编号}, % 可以留空: id={},
  % udc={UDC}, % 可以留空
  % catalognumber={分类号}, % 可以留空
  %
  %=========
  % 英文信息
  %=========
  % etitle={An Introduction to \LaTeX{} Thesis Template of Tsinghua University v\version},
  % 这块比较复杂,需要分情况讨论:
  % 1. 学术型硕士
  %    edegree:必须为Master of Arts或Master of Science(注意大小写)
  %             “哲学、文学、历史学、法学、教育学、艺术学门类,公共管理学科
  %              填写Master of Arts,其它填写Master of Science”
  %    emajor:“获得一级学科授权的学科填写一级学科名称,其它填写二级学科名称”
  % 2. 专业型硕士
  %    edegree:“填写专业学位英文名称全称”
  %    emajor:“工程硕士填写工程领域,其它专业学位不填写此项”
  % 3. 学术型博士
  %    edegree:Doctor of Philosophy(注意大小写)
  %    emajor:“获得一级学科授权的学科填写一级学科名称,其它填写二级学科名称”
  % 4. 专业型博士
  %    edegree:“填写专业学位英文名称全称”
  %    emajor:不填写此项
  % edegree={Doctor of Engineering},
  % emajor={Computer Science and Technology},
  % eauthor={Xue Ruini},
  % esupervisor={Professor Zheng Weimin},
  % eassosupervisor={Chen Wenguang},
  % 日期自动生成,若需指定按如下方式修改:
  % edate={December, 2005}
  %
  % 关键词用“英文逗号”分割
  ckeywords={漏洞, CVE, 互联网设备搜索引擎, 融合, 推送},
  ekeywords={Vulnerability, CVE, Internet-connected Devices Search Engine, Aggregation, Notification}
}

% 定义中英文摘要和关键字
\begin{cabstract}
  网络漏洞扫描工具是一种用于搜索网络主机中漏洞的工具,它依赖于各种已知漏洞组成的漏洞库。
  本文利用互联网设备搜索引擎,对其数据进行融合,作为主机的威胁情报,然后和CVE信息进行比对,
  提出了一种新的检测网络主机漏洞的方法,弥补了扫描工具漏洞库更新不及时等不足,并实现了这样的一个系统,
  能够监测目标主机上可能存在的威胁,在出现威胁时及时通知用户。我们将在文中讨论本系统的设计与实现中的细节。
\end{cabstract}

% 如果习惯关键字跟在摘要文字后面,可以用直接命令来设置,如下:
% \ckeywords{\TeX, \LaTeX, CJK, 模板, 论文}

\begin{eabstract}
  Network vulnerability scanners are tools that scan the network to find vulnerabilities on computers. 
  The scanning depends on known vulnerability database. In this paper, 
  we present a new way to find vulnerabilities that uses Internet-connected devices search engines. 
  It aggregates data from different Internet-connected devices search engines as threats information and then compares them with CVE details. 
  It fixes the disadvantage of scanners that vulnerability database is not always up-to-date. 
  We also implement this system that can monitor computers and notifies users whenever new vulnerabilities are found. 
  In this paper, we will discuss the design and implementation details of the system.
\end{eabstract}

% \ekeywords{\TeX, \LaTeX, CJK, template, thesis}
