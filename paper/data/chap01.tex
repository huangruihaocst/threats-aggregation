\chapter{引言}
\label{cha:intro}

\section{背景}
在互联网高度发达的21世纪,我们可以享用各种网络服务,比如浏览网页,收发邮件,文件共享等等;
互联网产业也在飞速发展,世界市值前十的公司里面有7家都是互联网公司,我们每天都在使用它们的产品,
给我们的日常生活带来了巨大的变化。但是,“机遇与风险并存”,在我们享用互联网带来的便利的同时,
互联网还带来了许多负面的问题,其中一个重中之重就是网络安全问题。

网络安全问题是在互联网中的计算机由于受到恶意的攻击,导致的软件或者硬件受到损害,
比如破坏、更改或者泄露所引起的问题。著名的网络安全问题就比如2017年5月12日开始的“永恒之蓝”——
全球范围内爆发的基于Windows网络共享协议进行攻击传播的蠕虫恶意代码的网络攻击事件,全世界有诸多高校内网、
大型企业内网和政府机构受到攻击,造成的损失不计其数。实际上,在该漏洞爆发之前,微软就已经发布了这个漏洞的补丁,
但是,用户并没有对自己的计算机上的漏洞有所了解以及重视,才会不幸受到黑客的攻击,如果能提前知道计算机上的漏洞就能未雨绸缪。

而为了解决这样的网络安全的问题,漏洞扫描工具应运而生。漏洞扫描工具是一种检测目标系统存在的漏洞和弱口令的工具,
可以监测目标中的通用漏洞。它们一般通过一系列逻辑判断,指出目标上的比如缓冲区溢出等漏洞。
著名的扫描工具有Metasploit,SQLMap,Nessus等。漏洞扫描工具不仅能扫描外网,因为是可以自己部署的,也可以扫描内网。

但是,既然是扫描漏洞,就必然需要一个漏洞库,现有的扫描工具都存在一个问题就是漏洞库更新不够快,
而且扫描出的漏洞也不是有已知编号的漏洞,常见的检测的漏洞比如SQL注入,XSS,CSRF等。
同时,漏洞扫描通常是进行渗透测试,可能会对系统产生不可预知的影响;因此进行扫描时,只能对已获得授权的目标进行扫描,不能对“别人”的主机进行扫描。

但是,实际上现在又有一种互联网设备搜索引擎,它们是用一些网络扫描工具,比如NMap,
扫描全网主机上的一些Fingerprint信息,然后提供给用户的搜索引擎。而这种搜索引擎每一家用的方法都不一样,
产生的结果也就不一样。因此,如果我们能对不同的互联网设备搜索引擎提供的数据进行融合,而不需要实际去扫描,
然后再和已知编号的漏洞进行对比,就能最大限度地发现目标主机上的潜在漏洞。在发现了这些潜在漏洞之后再由用户去实际判断是否真的受到这种威胁,
可以大大减轻安全维护的工作,对网络安全的研究有重要意义。


\section{项目需求}
\label{sec:requirements}

该项目的需求是寻找并实现一种发现主机威胁情报的方法,可以弥补现有扫描工具的一些不足,
如只能针对特定漏洞进行扫描,漏洞库不全,更新不及时,不够自动,需要授权,对目标可能产生不良影响等;
并且该系统也提供推送功能,在发现漏洞时推送威胁情报信息;也提供可视化工具,用户可以比较容易地操作该系统完成对主机系列的安全监测。

\section{本文贡献}
\label{sec:contributions} 

本文的贡献是基本满足了上述项目需求,设计并实现了一个威胁情报数据融合与推送服务系统。
这个系统的主要功能是监测一系列主机(域名、IP地址段等)的安全情况,
发现可能的威胁时及时地告知事先设置的邮箱或者其他联系方式。

该系统的主要特点是:
\begin{itemize}
  \item 威胁情报库完整并且更新非常及时,时刻和最新最全的CVE信息保持一致;
  \item 系统并不主动地对主机进行扫描和逻辑的判断,而是采用可以找到的互联网设备搜索引擎API并对其中的信息进行分析融合;
  \item 系统会尽可能多地获得主机系列中的全部主机,而对于每一台被发现的主机,系统会穷尽其相关信息并进行信息的融合;
  \item 主动发现威胁情报并通知,不需要操作者手动部署扫描;
  \item 系统提供了一个前端网站供直观地观察可能的威胁情报;
  \item 发现可能的威胁情报,但是并不实际检验是否真的受到该漏洞的威胁。
\end{itemize}

\section{本文结构}
\label{sec:structure}

本文共分为7章。

在第一章引言中介绍本文需求提出的背景和本文的工作;
第二章相关工作主要介绍目前现有的有关漏洞扫描的工作以及和本文的工作相关的知识背景,比如现有的扫描工具具体有哪些以及它们是如何工作的,
互联网设备搜索引擎是什么等,这些背景和后续本文的工作有很大的关系,可以说后续的工作都是以这些相关工作为背景的;
第三章系统设计着重介绍系统的整体结构以及每一部分从需求到分析到最终的设计,
不注重细节以及实际生产环境中的困难,而是高屋建瓴地介绍系统的结构与设计;第四章是第三章的后续,
从代码的角度介绍系统在实现过程中采用的方法与在实际生产环境中遇到的困难是如何克服的,以及如何提高效率和正确性;
第五章是结果检验,将介绍系统实现完成之后的特点、优势、使用方法,以及举例说明系统的正确性与用处;
第六章的总结与展望将对本文的工作进行总结,得出结论并提出本文工作的不足与可以进行改进的部分;
在最后一章致谢中将感谢为本文的完成提供帮助的人们。
