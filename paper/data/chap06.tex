\chapter{总结与展望}
\label{cha:conclusion}

本文设计并实现了一个威胁情报融合与推送系统,该系统能对市面上能见到的最主流的三个互联网设备搜索引擎进行数据的融合,
用这些数据和CVE的数据进行对比,然后找到主机中可能的高危漏洞。补足了现有的网络漏洞扫描工具漏洞库更新不及时,不能扫描未授权主机,
自动化程度低等问题。

不过,本系统仍然存在一些不足。比如,抓取的速度不够快,在进行数据融合的时候三个数据源实际没有依赖关系,但是并未采取并行的方法;
又如,主机数据和CVE信息进行比对的时候的算法没有任何先验知识,如果能运用一些先验知识或者机器学习的方法进行软件是否相同的判断可能会有更加准确的结果;
此外,现在对比主机和CVE的方法仅是通过软件和端口,并未使用其他指纹信息,如果能加入Shodan和Censys提供的更加详细的指纹信息,
运用自然语言处理的方法结合CVE的描述,会有更好的比对效果。