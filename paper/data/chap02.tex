\chapter{相关工作}
\label{cha:related-works}

\section{漏洞扫描工具}
\label{sec:scanners}

漏洞扫描是利用已知的漏洞数据库,对本地主机或远端主机进行脆弱性检验的一种方法,
是一种发现可利用漏洞的渗透测试,经常和防火墙和入侵检测系统结合使用以提高网络的安全性。
管理员可以根据漏洞扫描的结果来发现系统中的漏洞或者错误,在黑客攻击前做好防范的工作,避免不必要的损失。
和被动防御的防火墙和入侵检测系统相反,漏洞扫描是一种主动防御方式,可以提前发现问题,做到未雨绸缪。

而漏洞扫描工具就是可以进行漏洞扫描的工具,主要分为针对网络的、针对主机的和针对数据库的漏洞扫描工具,
而本文主要关注的是针对网络的漏洞扫描工具,意即通过网络来扫描远程计算机上的漏洞的工具。

而这之中比如Metasploit就是一款著名的网络漏洞扫描工具,它提供了数百个已知漏洞的利用工具,
可以发现目标主机上是否有相应的漏洞。在发现新漏洞时,其用户可以将漏洞添加到Metasploit的目录上,
任何使用该工具的人都可以用来检测特定系统是否受到该漏洞的威胁。Metasploit使用门槛低,易推广,
为漏洞检测的自动化和及时化做出了不可小觑的贡献。

其他地,也有SQLMap或者Netsparker这样专门针对SQL注入等特定的类型的漏洞进行扫描的工具,
还有像Nessus这样的全面的系统漏洞扫描工具和分析软件。这些扫描工具都以漏洞库为基础,
通过漏洞利用工具或者一系列逻辑判断,最终找到目标主机上的漏洞的工具。

\section{互联网设备搜索引擎}
\label{chap1:search-engines}

\subsection{Censys}
\label{sec:Censys}

Censys是一款互联网搜索引擎,它使用ZMap每天扫描整个IPv4地址空间,传言只需要45分钟就能对整个地址空间进行一次扫描,
然后搜集所有的互联网中的设备的信息,返回一份关于各种资源配置和部署信息的报告,而所谓资源就分为设备、网站或者证书。

Censys的官网上是这么介绍自己的:“Censys是一款给互联网安全从业人员提供的平台,它可以帮助发现、监视、和分析互联网中的设备。
我们有规律地探测所有的公共IP地址、热门域名,选择、管理和丰富这些搜索结果,然后通过一个搜索引擎和一套API让它变得易于理解。”

\subsection{Shodan}
\label{sec:Shodan}

Shodan经常被称为最可怕的搜索引擎,或者比Google更恐怖的搜索引擎。
它也是一款互联网设备搜索引擎,它可以搜索互联网中的各种设备,不仅仅是个人电脑,
甚至包括摄像头、打印机、路由器、红绿灯、粒子回旋加速器等等,只要是连接到互联网中的设备,
都有可能被Shodan发现。如果这些设备没有做好安全防御措施,就都有可能被Shodan随意进入。比如说默认密码,
如果在Shodan上搜索“Default Password”,会发现有不计其数的打印机的仍在使用默认用户名和密码,还有很多其他设备进入根本不需要认证,
只需要用浏览器访问即可进行连接。

在Shodan的网页上,用户可以搜索一个域名、IP地址段、甚至一种设备的名称,例如Webcam,
然后找到和关键词相关的互联网设备的信息,比如运行的服务、地理位置等。通过这些信息,
就可以分析找到互联网中的脆弱部分,侵犯用户的隐私与利益,或者作为管理员防患于未然,保护用户的利益。

\subsection{ZoomEye}
\label{sec:ZoomEye}

类似于Censys和Shodan,ZoomEye也是一款互联网设备搜索引擎,
而且是一款国产的互联网设备搜索引擎,中文名是钟馗之眼。
ZoomEye通过搜索引擎开放了他们的海量数据库,用户可以通过搜索得到主机的网站组件指纹
(如操作系统、Web应用、Web服务等)和主机设备指纹。不过,为了不被黑客大规模利用,
ZoomEye还没有完全开放,目前返回结果的条数受到了一定限制。

\section{CVE}
\label{sec:CVE}

CVE的全称是Common Vulnerabilities \& Exposures,
中文名是公共漏洞和暴露。CVE类似于一个字典表,给目前已知的漏洞一个统一的名字,
形如CVE-XXXX-XXXX,中间4位是年份,最后的几位(不一定是4位)表示实际的编号。
通过这个统一的名字,安全工作人员可以在各自的漏洞库和评估工具中共享数据,易于管理。即便如此,
不代表这些不同的工具就易于整合到一起,CVE只是提供了一个目录,或者“关键字”。

CVE的官方网站是https://cve.mitre.org,上面提供了每一个CVE的相关信息、统计信息,
用户也可以自行上传CVE或者下载CVE。而CVE Details是一个经过整理过的网站,
上面记载了所有CVE的详细信息,便于用户阅读,网址是https://www.cvedetails.com。

\section{Exploit Database}
\label{sec:Exploit-Database}

Exploit Database是一个漏洞库网站,上面有海量的漏洞利用脚本,
安全人员可以利用它来提高公司设备的安全性,网址是https://www.exploit-db.com。
其Exploits版块有Remote Exploits,用于展示最新的远程漏洞利用脚本;
Web Application Exploits,用来展示最新的Web应用漏洞利用脚本,比如SQL注入等;
Local \& Privilege Escalation Exploits等。除此之外,还有Shellcode,Papers等版块。
